% ============================
% Extra
% ============================
%
\begin{tikzpicture}[x=1mm, y=1mm]
    \tikzmath{\fs = \textsize + 0.5;}
    
    % Parchment paper
    \draw[fill=parchment] (0,0) rectangle (63,88);
    
    % Imagem
    \node[anchor=north west, inner sep=0,outer sep=0] (image) at (0, 88) {\includegraphics[width=63mm, height=88mm]{images/bugubi.png}};
    
    % Título
    \node[title, anchor=north west, minimum width=59mm](title) at (2,86) {Extra info - 1};
    
    % PARTY
    \node[fill=parchment, opacity=0.7, text opacity=1, draw opacity=1, inner sep=0mm, anchor=north west, font=\sffamily\fontsize{8}{8.5}, draw=black, rounded corners=0.4mm, fit={(2,24) (30,2)}] (PARTY) {};
    \node[tag, anchor=east] at ([xshift=-0.5mm] PARTY.north east) {PARTY};
    \node[fit={(2,24) (12,2)}, fill=black, opacity=0.9, text opacity=1, draw opacity=1, inner sep=0mm, font=\sffamily\fontsize{4.5}{6}\selectfont, draw=black, rounded corners=0.4mm, align=left] (players) {};
    \node[anchor=west, align=right, text width=8mm, font=\color{white}\sffamily\fontsize{5}{7}\selectfont] (partyp) at (PARTY.west) {%
    Lanza\\[0.9em]
    Marcelo\\[0.9em]
    Nestor\\[0.9em]
    Pará\\[-0.4em]\phantom{.}
    };
    
    \node[anchor=west, align=left, text width=30mm,font=\sffamily\fontsize{5}{7}\selectfont] (partyc) at ([xshift=-1mm] partyp.east) {%
    \textbf{Lanzara}\\
    \vspace{-0.9mm}{\fontsize{4}{8}\selectfont Nestorius, human mage}\\
    \textbf{Pipi}\\
    \vspace{-0.9mm}{\fontsize{4}{8}\selectfont Fairy thief}\\
    \textbf{Korvin}\\
    \vspace{-0.9mm}{\fontsize{4}{8}\selectfont Leonin barbarian}\\
    \textbf{Farpa}\\
    \vspace{-0.9mm}{\fontsize{4}{8}\selectfont Tabaxi monk}\\
    };
        
    % % PERS TRAITS
    % \node[fill=parchment, opacity=0.7, text opacity=1, draw opacity=1, inner sep=0mm, anchor=north west, font=\sffamily\fontsize{8}{8.5}, draw=black, rounded corners=0.4mm, fit={(29,24) (61,2)}] (PTRAITS) {};
    % \node[tag, anchor=east] at ([xshift=-0.5mm] PTRAITS.north east) {TEACHINGS};
    % \node[anchor=west, align=justify, text width=27mm, font=\sffamily\fontsize{4}{5}\selectfont] (traits) at (PTRAITS.west) {%
    % \faIcon[regular]{square}A disciplined mind brings happiness\\
    % \faIcon[regular]{square}If a man going down into a river is carried away by the current — how can he help others across?\\
    % \faIcon[regular]{square}You cannot travel the path until you have become the path itself.\\
    % \faIcon[regular]{square}There is no fear for one whose mind is not filled with desires\\
    % \faIcon[regular]{square}The root of suffering is attachment
    % };
    
    
    % BUGBEAR
    \node[fill=parchment, opacity=0.7, text opacity=1, draw opacity=1, inner sep=0mm, anchor=north west, font=\sffamily\fontsize{8}{8.5}, draw=black, rounded corners=0.4mm, fit={(2,80) (61,50)}] (BUGBEAR) {};
    \node[tag, anchor=west] at ([xshift=0.5mm] BUGBEAR.north west) {BUGBEAR};
    \node[anchor=north west, align=justify, yshift=-1mm, text width=27mm, font=\sffamily\fontsize{4.5}{5}\selectfont] (bugbear) at (BUGBEAR.north west) {%
    \textbf{Darkvision:} You can see in dim light within 60 feet of you as if it were bright light and in darkness as if it were dim light. You discern colors in that darkness as shades of gray.\\[0.5em]
    \textbf{Fey Ancestry:} You have advantage on saving throws you make to avoid or end the charmed condition on yourself.\\[0.5em]
    \textbf{Long-limbed:} When you make a melee attack on your turn, your reach for it is 5 feet greater than normal.\\[0.5em]
    };
    
    \node[anchor=north west, align=justify, yshift=-1mm, text width=28mm, font=\sffamily\fontsize{4.5}{5}\selectfont] (bugbear) at ([xshift=28mm] BUGBEAR.north west) {%
    \textbf{Powerful build:} You count as one size larger when determining your carrying capacity and the weight you can push, drag or lift.\\[0.5em]Carry: 255 kgs (17 * 7,5 * 2), push,drag or lift: 510 kgs\\[0.5em]
    \textbf{Sneaky:} You are proficient in the stealth skill. In addition, without squeezing, you can move through and stop in a space large enough for a small creature.\\[0.5em]
    \textbf{Surprise attack:} If you hit a creature with an attack roll, the creature takes an extra 2d6 damage if it hasn't taken a turn yet in the current combat.\\[0.5em]
    };
    
    % SOLDIER
    \node[fill=parchment, opacity=0.7, text opacity=1, draw opacity=1, inner sep=0mm, anchor=north west, font=\sffamily\fontsize{8}{8.5}, draw=black, rounded corners=0.4mm, fit={(2,48) (30,26)}] (SOLDIER) {};
    \node[tag, anchor=west] at ([xshift=0.5mm] SOLDIER.north west) {SOLDIER};
    \node[anchor=north west, align=justify, yshift=-1mm, text width=25mm, font=\sffamily\fontsize{4.5}{5}\selectfont] (soldier) at (SOLDIER.north west) {%
    \textbf{Savage Attacker: } You have trained to deal particularly damaging strikes. When you take the attack action and hit a target with a weapon as part of that action, you can roll the weapon's damage dice twice and use either roll against the target.\\[0.5em]You can use this benefit only once per turn.\\[0.5em]
    };

    % % FIGHTER
    % \node[fill=parchment, opacity=0.7, text opacity=1, draw opacity=1, inner sep=0mm, anchor=north west, font=\sffamily\fontsize{8}{8.5}, draw=black, rounded corners=0.4mm, fit={(2,35) (61,9)}] (FIGHTER) {};
    % \node[tag, anchor=west] at ([xshift=0.5mm] FIGHTER.north west) {FIGHTER};
    % \node[anchor=north west, align=justify, yshift=-1mm, text width=27mm, font=\sffamily\fontsize{4.5}{5}\selectfont] (bugbear) at (FIGHTER.north west) {%
    % \textbf{Great weapon fighting (Fighting Style):} You have honed your martial prowess and gain a Fighting style feat of your choice.\\[0.5em]When you roll a 1 or 2 on damage die for an attack you make with a melee weapon that you arewielding with two hands, you can reroll the die, and you must use the new roll.\\[0.5em]The weapon must have the Two-Handed or versatile property to gain this benefit.\\[0.5em]
    % };
    
    % \node[anchor=north west, align=justify, yshift=-1mm, text width=28mm, font=\sffamily\fontsize{4.5}{5}\selectfont] (bugbear) at ([xshift=28mm] FIGHTER.north west) {%
    % \textbf{Weapon mastery:} Your training with weapons allows you to use the mastery property of three kinds of simple or martial weapons of your choice.\\[0.5em]Whenever you finish a long rest, you can practice weapon drills and change one of those weapon choices.\\[0.5em]Weapons: Halberd, pike and hand axe\\[0.5em]
    % };

    % CHAMPION
    \node[fill=parchment, opacity=0.7, text opacity=1, draw opacity=1, inner sep=0mm, anchor=north west, font=\sffamily\fontsize{8}{8.5}, draw=black, rounded corners=0.4mm, fit={(31,48) (61,21)}] (FIGHTER) {};
    \node[tag, anchor=west] at ([xshift=0.5mm] FIGHTER.north west) {CHAMPION};
    
    \node[anchor=north west, align=justify, yshift=-1mm, text width=27mm, font=\sffamily\fontsize{4.5}{5}\selectfont] (bugbear) at (FIGHTER.north west) {%
    \textbf{Improved Critical:} Your attack rolls with weapons and Unarmed Strikes can score a critical hit on a roll of 19 or 20.\\[0.5em]
    \textbf{Remarkable Athlete:} Thanks to your athleticism, you have Advantage on Initiative rolls and Strength (Athletics) checks.\\[0.5em]In addition, when you make a running long jump, the distance you can cover increases by a number of feet equal to your Strength modifier.\\[0.5em]Long jump: 20ft (17 + 3)\\[0.5em]
    };

    % EMPTY
    % \node[fill=parchment, opacity=0.7, text opacity=1, draw opacity=1, inner sep=0mm, anchor=north west, font=\sffamily\fontsize{8}{8.5}, draw=black, rounded corners=0.4mm, fit={(31,19) (61,2)}] (FIGHTER) {};
    % \node[tag, anchor=west] at ([xshift=0.5mm] FIGHTER.north west) {NOTES};

    % Borda
    \draw[line width = 0.5mm, black] (0,0) rectangle (63,88);
    
    \end{tikzpicture}%%
    